\documentclass{beamer}

\setbeamercolor{blockstitle}{bg=yellow, fg=purple}
\setbeamercolor{blockstext}{bg=white, fg=teal}
\setbeamercovered{dynamic}
\usetheme{Madrid}
\usecolortheme{whale}

\section{Gale Shapeley Algorithm}
\section{Overlaying Concepts}
\subsection{Specifications}
\subsection[Examples]{Examples: Lists, Graphics, Tables}
\section[Sparkle]{Adding that Sparkle}
\subsection{Sections}
\subsection{Themes}
\section*{References}

\title[CS251 Project]{Seat Allocation Portal}
\subtitle{A CS251 Report by Group 31}
\date{Team 31 : !false}
\author[Prakash, Maloo \& Thakoor]
{%
  \texorpdfstring{
    \begin{columns}%[onlytextwidth]
      \column{.25\linewidth}
      \centering
      Pradyot Prakash\\
      {130050008}\\
      \href{mailto:pp2105@gmail.com}{pp2105@gmail.com}
      \column{.35\linewidth}
      \centering
      Chandra Maloo\\
      {130050009}\\
      \href{mailto:maloochandra@gmail.com}{maloochandra@gmail.com}
      \column{.35\linewidth}
      \centering
      Shantanu Thakoor\\
      {130050008}\\
      \href{mailto:shanu.thakoor@gmail.com}{shanu.thakoor@gmail.com}
    \end{columns}
  }
  {Pradyot Prakash, Chandra Maloo \& Shantanu Thakoor}
}
\institute{IIT Bombay}

\begin{document}

\frame{\maketitle}

\begin{frame}[t]{Introduction}{}
  \pause
  This is a software used for Allocation of seats of various engineering programmes to students based on their performance in a common entrance examination - JEE.
  The software takes the Merit list containing various ranks of all the students and their choices of preference of the programs. On the contrary to what it seems this not an easy task. We have used two algorithms for the same. First one is quite simple one assuming this to be an easy job. But as it turns out this is not a fair algorithm. So we have used the well known Gale Shapeley algorithm for a fair allocation of seats. \\
  \pause
  We have also created a web portal using Django framework for the same. On this portal each user has an account and he/she can login to fill in the preferences of his choice. Later on, when the allocation has been done then the student can see which college has been allotted to him. Based on previous year ranks we are also suggesting the possible programmes that a student might get admitted to based on previous year opening and closing ranks.
 \vspace{1cm}
 \end{frame}

\begin{frame}[t]{Seat Allocation Algorithms}
\pause
\textbf{Gale Shapeley Algorithm}  
\textbf{GaleShapely algorithm} is pathbreaking but simple in its conception and implementation.
Essentially, it goes as follows:
 \pause
\begin{itemize}
\item Every candidate applies to the first program on its preference list
 \pause
\item Every program rejects certain candidates and waitlists the others based on capacity and the meritlist it uses
 \pause
\item Rejected candidates move one index further along their preference list, waitlisted candidates stay on the same index
 \pause
\item Algorithm terminates when either all candidates reach the end of their preference list, or no candidate is rejected by any program when it filters its
applications
 \pause
\item When the algorithm terminates, a candidate is alloted the program that (s)he is currently waitlisted for; it (s)he is not waitlisted for anything, (s)he does not get alloted a seat
 \pause
\item Only minimal changes are needed to serve needs of DS and Foreign national student allocations
\end{itemize}
\end{frame}

\begin{frame}[t]{Seat Allocation Algorithms}
\pause
\textbf{MeritOrder algorithm} is also simple; perhaps a tad too simple. It follows a greedy algorithm approach, allotting candidates the best seat it possibly can the first time it sees them, and never looks at them again.
This algorithm is not fair to all the candidates
 \pause
\begin{itemize}
\item Arrange the candidates in order of their ranks
 \pause
\item Arrange the respective ranklists based on their category
 \pause
\item Start from the first Candidate
 \pause
\item For every candidate, go through his/her list of preferences in descending order until we find one that (s)he is eligible for; assign him/her that course
 \pause
\item If the candidate is not eligible for any course, (s)he is not alloted any seat
 \pause
\item Go to the next candidate
 \pause
\item Algorithm terminates when all the candidates are thus processed
 \pause
\item Only minimal changes are needed to serve needs of DS and Foreign national student allocations
\end{itemize}
\end{frame}

\begin{frame}[t]{Package Structure - code}
\pause
\textbf{code/common/} \\
This has the following common classes used by both the algorithms:  \\
\pause
\textbf{Constants.class}  \\
Contains all the constant variables names used anywhere in the code.  \\
For e.g. male="MALE", gen=0, etc. \\
\pause
\textbf{Candidates.class} \\
All the properties of a particular student are stored in an object of this class. It has all the attributes like rank, category, roll number, etc.  \\
\pause
\textbf{VirtualProgram.class} \\
All the programs in various colleges are stored in an object of this class. It has attributes which store the number of seats available in different categories for the program, course code, etc.\\
\pause
\textbf{MeritList.class}  \\
Contains the ranks of all the students of different categories\\
\end{frame}
\begin{frame}[t]{Package Structure - code}
\textbf{code/gale/} \\
\pause
\textbf{GaleShapeleyAdmission.class} \\
This class runs the entire GS algorithm and generates the corresponding output files\\
\pause
\textbf{code/merit/} \\
\textbf{MeritOrderAdmission.class}\\
This class runs the entire Merit Order algorithm and generates the corresponding output files\\
\pause
\textbf{code/main/} \\
\textbf{Main.class} \\
This class makes call to respective classes as and when needed and gets the entire allocation procedure executed\\
\end{frame}

\begin{frame}[t]{How to use}
\pause
\textbf{Allocation Part - Java}\\
\pause
Base folder(base/) contains the following: code, doc, test\_cases, Makefile\\
\pause
\textbf{Compiling}\\
\begin{itemize}[<+->]
\item Go to the base folder : \$ cd base/
\item Type into the command line : \$ make all
\end{itemize}
\pause
\textbf{Running the program}\\
\begin{itemize}[<+->]
\item Go to the base folder : \$ cd base/
\item Type into the command line, folder\_name is the name of the folder in test\_cases : \$ java code.Main.Main folder\_name
\item Here if you have various testcases with successive integers as folder names, e.g. 1 2 3 4 .. 100, then you can also run as :  \$ java Main.Main 1 100
\item Go to the folder : \$ cd test\_cases/folder\_name
\item GaleShapeleyAdmission.csv MeritOrderAdmission.csv GaleShapeleyAdmissionPretty.csv MeritOrderAdmissionPretty.csv are the outputs generated.
\end{itemize}
\end{frame}

\begin{frame}[t]{How to use}
\pause
\textbf{Adding your own testcases}\\
\begin{itemize}[<+->]
\item Go to the test\_cases folder : \$ cd base/test\_cases/
\item Create a directory : \$ mkdir folder\_name
\item Go the created directory : \$ cd folder\_name
\item Create the following three files : \$ touch ranklist.csv choices.csv programs.csv
\item Populate the above three files in the prescribed format (refer given testcases)
\end{itemize}
\pause
\textbf{Cleaning up...}\\
\begin{itemize}[<+->]
\item Go to the base folder : \$ base/
\item Type into the command line : \$ make clean
\item This removes all the .class files and also the java documentation, which were created when you executed the command `make all'
\end{itemize}
\end{frame}

\begin{frame}[t]{Parsing the pdf}
\textbf{update.py}\\
\begin{itemize}[<+->]
\item This file is present in the base directory of lab11. This preogram reads 'programmeCode.pdf' and generates the data\_u-2012.csv which contains the branch to their code mapping.
\item This program uses the shell script 'pdftotext' and generates the text output 'temp' from which the program further reads and populates the csv file.
\end{itemize}
\end{frame}

\begin{frame}[t]{How to use}
\textbf{Web Portal - Python}\\
\pause
\textbf{Pre-Allocation}\\
\begin{itemize}[<+->]
\item Student has to login into the portal using his roll number and the password (first time password is sent to his/her email id). In the submitted project, the password is same as the username. The user has the choice of changing it later.
\item There is also an option of forgot password where the user can reset his password by answering a security question
\item Once the student logs in he/she can change his/her personal details by clicking on 'Update profile', near 'Logout' button on top right. However, when the student logs in for the first time, (s)he is redirected to the 'Update profile' page so that (s)he can change whatever (s)he feels necessary.
\item He/she can fill in his/her choices of preference anytime by going to 'Update profile'
\end{itemize}
\end{frame}

\begin{frame}[t]{How to use}
\textbf{Web Portal - Python}\\
\pause
\textbf{Pre-Allocation contd..}\\
\begin{itemize}[<+->]
\item A student can also see the list of possible programs the students may get based on previous year opening and closing ranks using the 'Seat Predictor' which is the user's homepage.
\item The number of active users is displayed whenever a query is made in the homepage. The rank statistics are shown only when the number of users currently online are more than 50.
\end{itemize}
\end{frame}

\begin{frame}[t]{Django : Unchained}
\textbf{Structure}\\
\pause
The entire Lab11 has been built around the Python's Django framework. The main directory contains several files which are the different modules of the project. The Admission folder contains the basic files which render the entire project onto a browser.\\
\pause
\textbf {Admission folder}
\begin{itemize}[<+->]
\item settings.py : This file contains the configuration data for the entire website. It keeps track of the functions and middlewares to cal when a page is requested.
\item urls.py : Whenever any link is accessed, the control passes to this file. It contains the list of mappings of urls, i.e. which url represents which function in views.py
\item views.py : This is the main file which does all the background calculations and processes the data given by the user. The different functions in the file receive request from specific urls and return an appropriate page.
\end{itemize}
\end{frame}

\begin{frame}[t]{Django : Unchained}
\textbf{Structure contd..}\\
\begin{itemize}[<+->]
\item Data.py : It is a supporting file that helps render and keep track of the Dyanamic data being requested and accessed.
\item SessionMiddleware.py : This is a supporting class that clears the session data when there has been no activity for some specified time. This basically helps in increasing the security and reduce traffic on the site and thus help it run smoothly.
\end{itemize}

\textbf{Database folder}
\begin{itemize}[<+->]
\item models.py : This contains the Student class which is the template for the database. This stores all related information for the student.
\item migrations : This contains the history of all changes to database.
\end{itemize}
\end{frame}

\begin{frame}[t]{Django : Unchained}
\textbf{Structure contd..}\\
\textbf{HTML folder}
\begin{itemize}[<+->]
\item This folder contains the HTML pages which show the final webpages to the user. These contain the Python template tags and they are used to fill large data fields within some of the web pages.
\end{itemize}
\textbf{manage.py}
\begin{itemize}[<+->]
\item This is the most powerful file and is responsible for managing \textbf{Everything}. This function is used to create the Database, the entire project and then this controls the database as well. Any changes to the database can be made easily by this program.
\end{itemize}
Any other files present in the directory are the supplementary files for maintaining active users.
\end{frame}

\begin{frame}[t]{Django : Unchained}
\textbf{Structure contd..}\\
\textbf{Adding students to database}
\begin{itemize}[<+->]
\item We have populated the database with some default entries. You can add more users to the database though. Only these users will have access to the portal.
\item One example user is : G111100093 and password is the same.
\item The module reads from two files - 'ranklist.csv' and 'choices.csv'. To add new students, clear the two files and then add new users to the files. If they contain some users already then they will also get added to the database and that will not allow the already existing users to login.
\item After above task has been achieved, execute the populateDatabase() method in the Data.py file in Admissions directory.
\end{itemize}
\end{frame}

\begin{frame}[t]{How to use}
\textbf{Post-Allocation}\\
\begin{itemize}[<+->]
\item Once the last date for filling the preferences is over, the admin can get the choices.csv file from the database
\item Then the output can be put in test\_cases folder and the Allocation program can be run
\item The corresponding output can be put up then on the portal so that if the student logins then he can see his allocation
\end{itemize}
\pause
Come on, go and get your seat! We are awaiting you here :D
\end{frame}

\begin{frame}[t]{Team}
\textbf{Pradyot Prakash}\\
\pause
\begin{itemize}[<+->]
\item Django part (major)
\item Java part (minor)
\item Beamer
\item Video
\end{itemize}
  \pause
\textbf{Chandra Maloo}\\
\pause
\begin{itemize}[<+->]
\item Django part (major)
\item Java part (minor)
\item Testcases
\item LaTeX Report
\end{itemize}
\pause
\textbf{Shantanu Thakoor}\\
\pause
\begin{itemize}[<+->]
\item Java part (major)
\item Django part (minor)
\item Makefile
\item Javadoc
\end{itemize}
\pause
\end{frame}

\begin{frame}[t]{Bibliography}{}
     \vspace{10mm}
 \begin{beamerboxesrounded}[lower=blockstext,upper=blockstitle]{References}
    \pause
    \begin{itemize}[<+->]
    \item www.djangobook.com
    \item www.tangowithdjango.com
    \item www.stackoverflow.com
    \item www.oracle.com/technetwork/java/javase/documentation/javadoc-137458.html
    \item www.w3schools.com
 \end{itemize}
      \pause
      \bibliography{biblio}{}
\bibliographystyle{plain}
  \end{beamerboxesrounded}
\end{frame}

\end{document}
